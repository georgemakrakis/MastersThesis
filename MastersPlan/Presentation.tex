\documentclass[aspectratio=1610,pdftex,dvipsnames]{beamer}
\usetheme{Boadilla}
\usecolortheme{seahorse}
\beamertemplatenavigationsymbolsempty 
\addtobeamertemplate{footnote}{\hskip -2em}{}

\definecolor{BackGround}{RGB}{255,250,240}
%\setbeamercolor{background canvas}{bg=BackGround}


\usepackage{caption}
\captionsetup[figure]{labelformat=empty}

%packages and definitions
\usepackage{enumitem}
\setlist[itemize]{label=\textbullet}
\usepackage{amsmath,cancel,nicefrac}
\usepackage{ulem}
\usepackage{graphicx,animate}
%\usepackage{mypythonhighlight}% allows inclusion of graphics
\graphicspath{{../img/}} %path to graphics
\usepackage[yyyymmdd]{datetime} %date format
\renewcommand{\dateseparator}{.}

%%%%%
\usepackage{pgf,pgfplots}
\usepackage{tikz,grffile} % required for drawing custom shapes
\usetikzlibrary{shapes,arrows,automata,trees}
\pgfplotsset{compat=newest}
\usepgfplotslibrary{patchplots}
%%%%%

\usepackage{booktabs} % nice rules (thick lines) for tables
\usepackage{microtype} % improves typography for PDF

\usepackage[acronym,nomain,nonumberlist]{glossaries}
%\makenoidxglossaries

\usepackage{xcolor,colortbl} %change font color
\usepackage[numbers,sort&compress]{natbib} %use 'numbers' for numbered citations; 'round' for () instead [] for inline citations; nsf.bst
\usepackage{bibentry}
\usepackage[eulergreek]{sansmath}


\setlength{\bibsep}{0pt} %sets space between references
%\renewcommand{\bibsection}{} %suppresses large 'references' heading
\renewcommand\bibpreamble{\vspace{-0.2\baselineskip}} %sets spacing after heading if not using default references heading

%%%%% user commands
\newcommand\blfootnote[1]{%
  \begingroup
  \renewcommand\thefootnote{}\footnote{#1}%
  \addtocounter{footnote}{-1}%
  \endgroup
}


\makeatletter
\renewcommand{\@biblabel}[1]{#1.\hfill} %bibliography ordered list has numbers left flush
\makeatother

\newcommand{\edit}[1]{\textcolor{blue}{#1}} %shortcut for changing font color on revised text
\newcommand{\fn}[1]{\footnote{#1}} %shortcut for footnote tag
\newcommand*\sq{\mathbin{\vcenter{\hbox{\rule{.3ex}{.3ex}}}}} %makes a small square as a separator $\sq$
\newcommand{\sk}[1]{\sout{#1}} %shortcut for strikethrough
\newcommand{\x}{\cellcolor{lightgray}\textbf{X}} %use to shade in table cell

\newcommand{\acf}{\acrfull} %full acronym
\newcommand{\acl}{\acrlong} %long acronym
\newcommand{\acs}{\acrshort} %short acronym
\newcommand{\acfp}{\acrfullpl} %full acronym plural
\newcommand{\aclp}{\acrlongpl} %long acronym plural
\newcommand{\acsp}{\acrshortpl} %short acronym plural
\newcommand{\Acf}{\Acrfull} %full acronym first letter capital
\newcommand{\Acl}{\Acrlong} %long acronym first letter capital

\newacronym{pid}{PID}{Proportional-Integral-Derivative}
\newacronym{npp}{NPP}{Nuclear Power Plant}
\newacronym{msnb}{MSNB}{Molten Salt Nuclear Battery}
\newacronym{msr}{MSR}{Molten Salt Reactor}
\newacronym{msre}{MSRE}{Molten Salt Reactor Experiment}
\newacronym{lwr}{LWR}{Light Water Reactor}
\newacronym{smr}{SMR}{Small Modular Reactor}
\newacronym{nrc}{NRC}{Nuclear Regulatory Commission}
\newacronym{ans}{ANS}{American Nuclear Society}
\newacronym{inl}{INL}{Idaho National Laboratory}
\newacronym{nrel}{NREL}{National Renewable Energy Laboratory}
\newacronym{oak}{ORNL}{Oak Ridge National Laboratory}
\newacronym{lti}{LTI}{Linear Time-Invariant}
\newacronym{mm}{MIMO}{Multi-Input Multi-Output}
\newacronym{ss}{SISO}{Single-Input Single-Output}

\newcommand{\UF}[1][4]{$UF_{#1}$}
\newcommand{\flinak}{$FLiNaK$ }

\newcommand{\B}[1][]{$^{#1}B$ }
\newcommand{\Be}[1][]{$^{#1}Be$ }
\newcommand{\I}[1][135]{$^{#1}I$ }
\newcommand{\Xe}[1][135]{$^{#1}Xe$ }
\newcommand{\Nd}[1][149]{$^{#1}Nd$}
\newcommand{\Pm}[1][149]{$^{#1}Pm$ }
\newcommand{\Sa}[1][149]{$^{#1}Sa$ }
\newcommand{\Gd}[1][157]{$^{#1}Gd$ }
\newcommand{\U}[1][]{$^{#1}U$ }
\newcommand{\Pu}[1][239]{$^{#1}Pu$ }
\newcommand{\Ca}[1][]{$^{#1}Ca$ }
\newcommand{\Am}[1][]{$^{#1}Am$ }
\newcommand{\Po}[1][]{$^{#1}Po$ }
\newcommand{\Ra}[1][]{$^{#1}Ra$ }
%%%%%

%%%%%



\title[\acs{pid} Controller Design: \acs{msnb}]{Design of a \acs{pid} Controller for a Molten Salt Microreactor}
\subtitle{Master's Plan}
\author[Root]{Sam J. Root,\textsuperscript{1}\\
    Major Professor: Michael McKellar,\textsuperscript{1}\\
    Committee Members: Robert A. Borrelli\textsuperscript{1}, 
    Dakota Roberson\textsuperscript{2}
    }
\institute[Idaho Falls Center]{\vspace{0.5cm}\\
    University of Idaho $\sq$ Idaho Falls Center for Higher Education\\
    \textsuperscript{1}Department of Nuclear Engineering and Industrial Management\\
    \textsuperscript{2}Department of Electrical and Computer Engineering\\
    \vspace{0.10in}
    %\includegraphics[width=0.20\textwidth]{ne-logo.jpg}
    }

\date{2022.10.13}
    

\AtBeginSection[]{
    \begin{frame}
        \vfill
        \centering
        \begin{beamercolorbox}[sep=8pt,center,shadow=true,rounded=true]{title}
            \usebeamerfont{title}\insertsectionhead\par
        \end{beamercolorbox}
        
        \vfill
        
    \end{frame}
}

\begin{document}
    \nobibliography*
{    \setbeamertemplate{footline}{} 
    \begin{frame}
        \titlepage
    \end{frame}
} 

\begin{frame}{Outline}
    \tableofcontents
\end{frame}

\section{Scope}
\begin{frame}{\acf{msnb}}
\begin{columns}
    %Column 1
    \begin{column}{0.5\textwidth}
        \begin{itemize}
            \item Self-Contained liquid fueled molten salt micro-reactor - 10 year design
            \item 1 MW design using HALEU \UF  dissolved in \flinak
            \item Criticality is manipulated using axial control drums \begin{itemize}
                \item Neutron absorber plate covering cylinders of neutron reflector
                \item Drums are rotated to point more absorber towards the core to insert negative control reactivity
            \end{itemize}
        \end{itemize}
    \end{column}
    %Column 2
    \begin{column}{0.5\textwidth}
        \only<1-2>{
        \begin{figure}[!ht]
            \centering
            \begin{tikzpicture}
    %Core
    \draw node at (1.5,1.5) {Core};
    \draw[red, very thick] (0,0) rectangle (3,3);
    \filldraw[red,opacity=0.2] (0,0) rectangle (3,3) ;
    %Riser/Chimney
    \draw[->] (1.5,3) -- (1.5,3.5);
    %HEX
    \draw node at (1.5,4) {Heat Exchanger};
    \draw[blue, very thick] (0,3.5) rectangle (3,4.5);
    \filldraw[blue,opacity=0.2] (0,3.5) rectangle (3,4.5) ;
    %Downcomer
    \draw[->] (3,4) -- (3.5,4) -- (3.5,-0.5)  -- (1.5,-0.5);
    \draw[->] (0,4) -- (-0.5,4) -- (-0.5,-0.5)  -- (1.5,-0.5);
    \draw[->] (1.5,-0.5) -- (1.5,0);
\end{tikzpicture}

            \caption{Simplified schematic drawing of an \acs{msnb}}
            \label{fig:tikz_msnb}
        \end{figure}
        }
        \only<3>{
            \begin{figure}
                \animategraphics[width=0.8\textwidth, controls, loop, trim=12cm 1.5cm 2cm 2.5cm]{1}{../img/Plotter/XY-}{0}{12}
                \caption{MsNB Control Drums}
            \end{figure}
        }
    \end{column}
\end{columns}
\end{frame}

\begin{frame}{Background on \acs{msnb}}
    \begin{block}{Neutronics \cite{PetersonMS}}
        \only<1>{
        \begin{itemize}
            \item Control drums give a uniform axial and radial flux profile for all reactivity insertions
            \item Fission product poisoning is the biggest challenge to reach the desired 10-year lifespan
            \item Control drum vs. reactivity curve is sinusoidal
        \end{itemize}    
        }
    \end{block}
    
    \begin{block}{Thermal Hydraulics \cite{CarterPHD}}
        \only<2>{
            \begin{itemize}
                \item The counteracting passive feedback effects of temperature reactivity and flow reactivity produce stable autonomous load following for relatively small ramp function power demand transients
                \item An in-core helix device can be used to manipulate temperature and power profiles in the core, as well as minimize advective loss of delayed-neutron precursors
            \end{itemize}    
            }
    \end{block}
    
    \begin{block}{Process Control}
        \only<3>{
            \begin{itemize}
                \item Design controller \textit{compliment} the autonomous capabilities provided by the passive feedback mechanisms
                \item Allow larger faster, more aggressive power changes
                \item Additional focus on time periods with a high degree of time variance (start-up, shut-down, re-start)
            \end{itemize}    
            }
    \end{block}

    \blfootnote{\tiny\cite{PetersonMS} \tiny\bibentry{PetersonMS}}
    \blfootnote{\tiny\cite{CarterPHD} \tiny\bibentry{CarterPHD}}
\end{frame}

\section{Applied Literature Review}

\begin{frame}{Passive Feedback}{\only<2>{Temperature Reactivity}\only<3>{Flow Reactivity}}
   \only<1>{
    \begin{figure}[!ht]
        \centering
        
\begin{tikzpicture}
    
    %Sum
    \draw[->] (-3,0) node[anchor=east]{$\dot{Q}_{HEX}(s)$}-- (-2,0);
    \draw (-1.75,0) circle (0.25)node{\scriptsize$+$};
    %Core/Main
    \draw[->](-1.5,0) -- (0,0);
    \draw (0,-0.5) rectangle (4,0.5) node[pos=0.5]{Core};
    \draw[->] (4,0) -- (7,0) node[anchor=west]{$\dot{Q}_{Core}(s)$};
    %Temperature Feedback
    \draw[->] (5.5,0) -- (5.5,-1.5) -- (4,-1.5);
    \draw (0,-2) rectangle (4,-1) node[pos=0.5]{Temperature Feedback} ;
    \draw[->] (0,-1.5) -- (-1.75,-1.5) -- (-1.75,-0.25);
    %Flow Feedback
    \draw[->] (5.5,0) -- (5.5,1.5) -- (4,1.5);
    \draw (0,2) rectangle (4,1) node[pos=0.5]{Flow Feedback} ;
    \draw[->] (0,1.5) -- (-1.75,1.5) -- (-1.75,0.25);
\end{tikzpicture}

        \caption{Simplified block diagram of two primary passive feedback mechanisms in an \acs{msnb}}
        \label{fig:tikz_passive}
    \end{figure}
    }
    \only<2>{
        \begin{itemize}
            \item Doppler broadening of resonance peaks results in more epithermal neutrons absorbed by \U[238] etc. \cite[Ch. 6]{DH}
            \item Molten salt fuels have high thermal expansion coefficient \cite{PetersonMS}
            \item Increased temperature leads to lower heavy metal density and smaller macroscopic fission cross-section at high temperature
            \item Similar to moderator thinning in \acsp{lwr}
            \item These two effects combine to result in less power production at higher temperature
        \end{itemize}

        \blfootnote{\tiny\cite{DH} \tiny\bibentry{DH}}
        \blfootnote{\tiny\cite{PetersonMS} \tiny\bibentry{PetersonMS}}
    }
    \only<3>{
        \begin{itemize}
            \item Driven by advection of delayed neutron precursors \cite[Ch. 3]{Kerlin}
            \begin{itemize}
                \item Most fission events release daughter neutrons \textit{promptly}
                \item Sometimes, unstable nuclides which decay by neutron emission are produced instead
                \item $t_{\nicefrac{1}{2}}$ from less than a second to over a minute \cite[Ch. 6]{Lamarsh}
            \end{itemize}
            \item Precursors produced near the core exit and long lived precursors may emit their neutrons outside of the core, so they are effectively \textit{lost} from the fission chain reaction
            \item In natural circulation, larger power transport requires a higher flow rate, and greater delayed neutron losses
        \end{itemize}

        \blfootnote{\tiny\cite{Kerlin} \tiny\bibentry{Kerlin}}
        \blfootnote{\tiny\cite{Lamarsh} \tiny\bibentry{Lamarsh}}
    }
\end{frame}

\begin{frame}{Main Operational Control Problem - Transport Delay}{\only<3>{Immediate Response}\only<4>{Heat Exchanger Perturbation}\only<5>{Core Perturbation}}
    \only<1>{
        \begin{itemize}
            \item Start design process by discussing dynamics associated with anticipated transients
            \begin{itemize}
                \item Natural circulation flow mode
                \item Passive feedback mechanisms
                \item Transport delays separating heat exchanger and core
            \end{itemize}
            \item Thought Experiment
            \begin{itemize}
                \item Step increase in power demand to a steady-state critical \acs{msnb}
                \item Set-point is instantaneously equal to heat exchanger power consumption
                \item Ideal controller which produces rapid load following with minimal overshoot
            \end{itemize}
        \end{itemize}
    }
    \only<2>{
        \begin{figure}[!ht]
            \centering
            \input{../tikz/thoughtexperiment-1}
            \caption{Simplified schematic drawing of a natural circulation \acs{msnb}}
            \label{fig:tikz_thoughtexperiment-1}
        \end{figure}
    }
    \only<3>{
        \begin{figure}[!ht]
            \centering
            \input{../tikz/thoughtexperiment-2}
            \caption{Simplified schematic drawing of a natural circulation \acs{msnb}}
            \label{fig:tikz_thoughtexperiment-2}
        \end{figure}
    }
    \only<4>{
        \begin{figure}[!ht]
            \centering
            \begin{tikzpicture}
    %Core
    \draw node at (1.5,1.5) {Core};
    \draw[red, very thick] (0,0) rectangle (3,3);
    \filldraw[red,opacity=0.2] (0,0) rectangle (3,3) ;
    %Riser/Chimney
    \draw[->] (1.5,3) -- (1.5,4) -- (4,4); \draw[->] (1.5,3) -- (1.5,4) -- (4,4);
    \draw[->,ultra thick, red] (1.4,3) -- (1.4,4.1) -- (4,4.1);
    %HEX
    \draw node at (5.5,4) {Heat Exchanger};
    \draw[blue, very thick] (4,3.5) rectangle (7,4.5);
    \filldraw[blue,opacity=0.2] (4,3.5) rectangle (7,4.5) ;
    %Downcomer
    \draw[->] (5.5,3.5) -- (5.5,-1) -- (1.5,-1) -- (1.5,-1) -- (1.5,0);
    \draw[->,ultra thick, blue] (5.6,3.5) -- (5.6,0);
\end{tikzpicture}

            \caption{Simplified schematic drawing of a natural circulation \acs{msnb}}
            \label{fig:tikz_thoughtexperiment-3}
        \end{figure}
    }
    \only<5>{
        \begin{figure}[!ht]
            \centering
            \input{../tikz/thoughtexperiment-4}
            \caption{Simplified schematic drawing of a natural circulation \acs{msnb}}
            \label{fig:tikz_thoughtexperiment-4}
        \end{figure}
    }
\end{frame}

\begin{frame}{Design Decisions}{\only<1>{Pre-Filter}\only<2>{Dead-band}\only<3>{Disturbance Feedforward}}
        \only<1>{
        \begin{columns}
            \begin{column}{0.5\textwidth}
                \begin{figure}
                    \resizebox{0.95\textwidth}{!}{% This file was created by matlab2tikz.
%
%The latest updates can be retrieved from
%  http://www.mathworks.com/matlabcentral/fileexchange/22022-matlab2tikz-matlab2tikz
%where you can also make suggestions and rate matlab2tikz.
%
\definecolor{mycolor1}{rgb}{0.00000,0.44700,0.74100}%
\definecolor{mycolor2}{rgb}{0.85000,0.32500,0.09800}%
%
\begin{tikzpicture}

\begin{axis}[%
width=4.521in,
height=3.566in,
at={(0.758in,0.481in)},
scale only axis,
xmin=0,
xmax=10,
xtick={ 0,  1,  2,  3,  4,  5,  6,  7,  8,  9, 10},
xticklabel style={font=\sansmath\sffamily},
xlabel style={font=\sansmath\sffamily},
xlabel={Time Constants, $\tau_F$},
ymin=-0.05,
ymax=1.05,
ylabel style={font=\sansmath\sffamily},
ylabel={Normalized Magnitude},
yticklabel style={font=\sansmath\sffamily},
ytick={0,0.632,0.865,0.95,1},
axis background/.style={fill=white},
legend style={at={(0.711,0.171)}, anchor=south west, legend cell align=left, align=left, font=\sansmath\sffamily}
]

\addplot [color=mycolor1]
  table[row sep=crcr]{%
0	0\\
0.2	0\\
0.4	0\\
0.6	0\\
0.8	0\\
0.999999999999993	0\\
1	1\\
1.00000000000001	1\\
1.20000000000001	1\\
1.40000000000001	1\\
1.60000000000001	1\\
1.80000000000001	1\\
2.00000000000001	1\\
2.20000000000001	1\\
2.40000000000001	1\\
2.60000000000001	1\\
2.80000000000001	1\\
3.00000000000002	1\\
3.20000000000002	1\\
3.40000000000002	1\\
3.60000000000002	1\\
3.80000000000002	1\\
4.00000000000002	1\\
4.20000000000002	1\\
4.40000000000002	1\\
4.60000000000002	1\\
4.80000000000002	1\\
4.99999999999996	1\\
5.00000000000002	1\\
5.20000000000002	1\\
5.40000000000002	1\\
5.60000000000002	1\\
5.80000000000002	1\\
6.00000000000002	1\\
6.20000000000002	1\\
6.40000000000002	1\\
6.60000000000002	1\\
6.80000000000002	1\\
7.00000000000002	1\\
7.20000000000002	1\\
7.40000000000002	1\\
7.60000000000002	1\\
7.80000000000002	1\\
8.00000000000002	1\\
8.20000000000002	1\\
8.40000000000002	1\\
8.60000000000002	1\\
8.80000000000002	1\\
9.00000000000002	1\\
9.20000000000002	1\\
9.40000000000001	1\\
9.60000000000001	1\\
9.80000000000001	1\\
10	1\\
};
\addlegendentry{Input}

\addplot [color=mycolor2, dashed]
  table[row sep=crcr]{%
0	0\\
0.2	0\\
0.4	0\\
0.6	0\\
0.8	0\\
0.999999999999993	0\\
1	0\\
1.00000000000001	1.42108547152019e-14\\
1.20000000000001	0.181269226666678\\
1.40000000000001	0.329679920797011\\
1.60000000000001	0.451188323173276\\
1.80000000000001	0.550670991417293\\
2.00000000000001	0.63212051332198\\
2.20000000000001	0.698805743378635\\
2.40000000000001	0.753402993352831\\
2.60000000000001	0.798103442046079\\
2.80000000000001	0.834701074973048\\
3.00000000000002	0.864664683281515\\
3.20000000000002	0.889196811483763\\
3.40000000000002	0.909282019778302\\
3.60000000000002	0.925726397897851\\
3.80000000000002	0.939189916312655\\
4.00000000000002	0.950212913156196\\
4.20000000000002	0.959237779886358\\
4.40000000000002	0.966626716003575\\
4.60000000000002	0.972676265384934\\
4.80000000000002	0.977629217628252\\
4.99999999999996	0.981684352048706\\
5.00000000000002	0.981684352048707\\
5.20000000000002	0.985004415388737\\
5.40000000000002	0.987722653414635\\
5.60000000000002	0.989948158535683\\
5.80000000000002	0.991770248064495\\
6.00000000000002	0.993262048833503\\
6.20000000000002	0.994483432030772\\
6.40000000000002	0.995483416040408\\
6.60000000000002	0.996302133721938\\
6.80000000000002	0.996972443082479\\
7.00000000000002	0.997521245983607\\
7.20000000000002	0.997970567807256\\
7.40000000000002	0.998338441411407\\
7.60000000000002	0.998639630851823\\
7.80000000000002	0.998886223915294\\
8.00000000000002	0.999088117244848\\
8.20000000000002	0.999253413526685\\
8.40000000000002	0.999388746679343\\
8.60000000000002	0.999499548096076\\
8.80000000000002	0.999590264625684\\
9.00000000000002	0.999664537040124\\
9.20000000000002	0.999725346151436\\
9.40000000000001	0.999775132442167\\
9.60000000000001	0.999815894010477\\
9.80000000000001	0.999849266760823\\
10	0.999876590058521\\
};
\addlegendentry{Output}

\end{axis}

\begin{axis}[%
width=5.833in,
height=4.375in,
at={(0in,0in)},
scale only axis,
xmin=0,
xmax=1,
ymin=0,
ymax=1,
axis line style={draw=none},
ticks=none,
axis x line*=bottom,
axis y line*=left
]
\end{axis}
\end{tikzpicture}%}
                    \caption{Pre-Filter on a step-function}
                \end{figure}
            \end{column}
            \begin{column}{0.5\textwidth}
                \begin{figure}
                    \resizebox{0.95\textwidth}{!}{% This file was created by matlab2tikz.
%
%The latest updates can be retrieved from
%  http://www.mathworks.com/matlabcentral/fileexchange/22022-matlab2tikz-matlab2tikz
%where you can also make suggestions and rate matlab2tikz.
%
\definecolor{mycolor1}{rgb}{0.00000,0.44700,0.74100}%
\definecolor{mycolor2}{rgb}{0.85000,0.32500,0.09800}%
%
\begin{tikzpicture}

\begin{axis}[%
width=4.521in,
height=3.566in,
at={(0.758in,0.481in)},
scale only axis,
xmin=0,
xmax=10,
xtick={ 0,  1,  2,  3,  4,  5,  6,  7,  8,  9, 10},
xlabel style={font=\color{white!15!black}},
xlabel={Time},
ymin=-0.1,
ymax=1.1,
ylabel style={font=\color{white!15!black}},
ylabel={Value},
axis background/.style={fill=white},
legend style={at={(0.711,0.171)}, anchor=south west, legend cell align=left, align=left, draw=white!15!black}
]
\addplot [color=mycolor1]
  table[row sep=crcr]{%
0	0\\
0.2	0\\
0.4	0\\
0.6	0\\
0.8	0\\
0.999999999999993	0\\
1	0\\
1.00000000000001	3.5527136788005e-15\\
1.20000000000001	0.0500000000000035\\
1.40000000000001	0.100000000000004\\
1.60000000000001	0.150000000000004\\
1.80000000000001	0.200000000000004\\
2.00000000000001	0.250000000000004\\
2.20000000000001	0.300000000000004\\
2.40000000000001	0.350000000000004\\
2.60000000000001	0.400000000000004\\
2.80000000000001	0.450000000000004\\
3.00000000000002	0.500000000000004\\
3.20000000000002	0.550000000000004\\
3.40000000000002	0.600000000000004\\
3.60000000000002	0.650000000000004\\
3.80000000000002	0.700000000000004\\
4.00000000000002	0.750000000000004\\
4.20000000000002	0.800000000000004\\
4.40000000000002	0.850000000000004\\
4.60000000000002	0.900000000000004\\
4.80000000000002	0.950000000000004\\
4.99999999999996	0.99999999999999\\
5.00000000000002	1\\
5.20000000000002	1\\
5.40000000000002	1\\
5.60000000000002	1\\
5.80000000000002	1\\
6.00000000000002	1\\
6.20000000000002	1\\
6.40000000000002	1\\
6.60000000000002	1\\
6.80000000000002	1\\
7.00000000000002	1\\
7.20000000000002	1\\
7.40000000000002	1\\
7.60000000000002	1\\
7.80000000000002	1\\
8.00000000000002	1\\
8.20000000000002	1\\
8.40000000000002	1\\
8.60000000000002	1\\
8.80000000000002	1\\
9.00000000000002	1\\
9.20000000000002	1\\
9.40000000000001	1\\
9.60000000000001	1\\
9.80000000000001	1\\
10	1\\
};
\addlegendentry{Input}

\addplot [color=mycolor2, dashed]
  table[row sep=crcr]{%
0	0\\
0.2	0\\
0.4	0\\
0.6	0\\
0.8	0\\
0.999999999999993	0\\
1	0\\
1.00000000000001	2.52378540224252e-29\\
1.20000000000001	0.00468269333333398\\
1.40000000000001	0.0175800198007507\\
1.60000000000001	0.0372029192066846\\
1.80000000000001	0.0623322521456803\\
2.00000000000001	0.0919698716695085\\
2.20000000000001	0.125298564155345\\
2.40000000000001	0.161649251661796\\
2.60000000000001	0.200474139488484\\
2.80000000000001	0.241324731256742\\
3.00000000000002	0.283833829179625\\
3.20000000000002	0.327700797129063\\
3.40000000000002	0.372679495055428\\
3.60000000000002	0.418568400525541\\
3.80000000000002	0.46520252092184\\
4.00000000000002	0.512446771710955\\
4.20000000000002	0.560190555028414\\
4.40000000000002	0.60834332099911\\
4.60000000000002	0.656830933653771\\
4.80000000000002	0.705592695592941\\
4.99999999999996	0.754578911987813\\
5.00000000000002	0.754578911987827\\
5.20000000000002	0.799066202819486\\
5.40000000000002	0.835489316845594\\
5.60000000000002	0.865310041159399\\
5.80000000000002	0.8897251858382\\
6.00000000000002	0.90971461612212\\
6.20000000000002	0.926080577836966\\
6.40000000000002	0.939479894328106\\
6.60000000000002	0.950450327081035\\
6.80000000000002	0.959432157972642\\
7.00000000000002	0.966785859324477\\
7.20000000000002	0.972806560919127\\
7.40000000000002	0.977735894591724\\
7.60000000000002	0.981771691761507\\
7.80000000000002	0.98507592309934\\
8.00000000000002	0.987781198977837\\
8.20000000000002	0.989996091589918\\
8.40000000000002	0.991809492331058\\
8.60000000000002	0.993294179322215\\
8.80000000000002	0.994509738250642\\
9.00000000000002	0.995504953752146\\
9.20000000000002	0.996319767309325\\
9.40000000000001	0.996986880243117\\
9.60000000000001	0.997533066131301\\
9.80000000000001	0.997980245325918\\
10	0.998346364693745\\
};
\addlegendentry{Output}

\end{axis}

\begin{axis}[%
width=5.833in,
height=4.375in,
at={(0in,0in)},
scale only axis,
xmin=0,
xmax=1,
ymin=0,
ymax=1,
axis line style={draw=none},
ticks=none,
axis x line*=bottom,
axis y line*=left
]
\end{axis}
\end{tikzpicture}%}
                    \caption{Pre-Filter on a ramp-function}
                \end{figure}
            \end{column}
        \end{columns}
        }

        \only<2>{
            \begin{itemize}
                \item Minor perturbations will cause power fluctuations
                \item Fine grain and constant control actuation would burn out servos prematurely
                \item When the error is small, allow the passive feedback mechanisms to fine tune
            \end{itemize}
        }

        \only<3>{
            \begin{itemize}
                \item Disturbances, particularly temperature reactivity, are extremely high frequency - On order of mean neutron lifetime \cite[Ch. 7]{Lamarsh}
                \item Control actuation is similarly quick
                \item Time delays on the order of dozens of seconds to minutes
                \item Inserting control reactivity to counteract temperature reactivity at the exact right moment would be difficult or impossible, so it is unlikely that a disturbance feedforward controller could reject the disturbance before it causes error \cite[Ch. 10]{Bequette}
                \item Disturbance rejection best left to feedback \acs{pid} controller
            \end{itemize}
            
            \blfootnote{\tiny\cite{Lamarsh} \tiny\bibentry{Lamarsh}}
            \blfootnote{\tiny\cite{Bequette} \tiny\bibentry{Bequette}}
        }

\end{frame}

\begin{frame}{Time-Variance and Non-Linearity}{\only<1>{Operational Control}\only<2>{Special Cases}}
    \only<1>{\begin{itemize}
        \item Sinusoidal control drum angle vs. reactivity curve - non-linear reactivity actuation - Taylor Series approximation to linearize around the operating point \cite[Ch. 2]{Bequette}
        \item Fissile depletion changes the amount of control reactivity required to make the core critical - time-variant controller bias - Bias and gain schedule \cite{GainSchedule}
        \item Control drums manipulate \textit{criticality}, not power directly - highly time-dependent control actuation
    \end{itemize}

    \blfootnote{\tiny\cite{GainSchedule} \tiny\bibentry{GainSchedule}}
    \blfootnote{\tiny\cite{Bequette} \tiny\bibentry{Bequette}}
    }
    \only<2>{\begin{itemize}
        \item Equilibrium Poisoning
        \begin{itemize}
            \item Poisons like \Xe and \Sa build-up after the reactor starts - this causes a negative reactivity insertion
            \item Reach equilibrium over the first \~100 hours
            \item Gain/Bias schedule may be used
            \item Alternatively, a burnable poison with appropriate \textit{effective} half-life could be selected - \Gd[157] shows promise to counteract \Xe build-up
        \end{itemize}
        \item Restart Poisoning
        \begin{itemize}
            \item \Xe levels increase following shut-down because its beta-precursor (\I) decays faster
            \item Requires a lot of excess control reactivity and very good control to counteract the positive reactivity insertion of poison burn-out after restart
            \item  Low-flux burn-out instead of full shut-down
            \item Stripping \Xe and other fission gasses before re-start
        \end{itemize}
    \end{itemize}
    }

\end{frame}

\section{Future Work}

\begin{frame}{Control Drum Characterization}{MCNP}
    \begin{itemize}
        \item Use KCODE to develop control drum vs. reactivity curve at various points in the core lifespan
        \item Use Burn-up routine to study how the core criticality at different conditions effects the control drum vs. reactivity
        \begin{itemize}
            \item Cold/clean start-up
            \item Burnable poison start-up
            \item Equilibrium poisoning
            \item Long-term depletion of fissile isotopes
        \end{itemize} 
        \item Develop bias/unity point schedule 
        \item Will to use HPC or Falcon
    \end{itemize}
\end{frame}

\begin{frame}{Process Simulation}{Python}
    \begin{itemize}
        \item 1D$+$time finite element model that accounts for passive feedback mechanisms during unsteady state subcritical, critical, and supercritical modes to calculate the core power and flow loop temperature profile over time
        \item Simulate system response to:
        \begin{itemize}
            \item Control actuation
            \item Heat exchanger transients
        \end{itemize}
        \item Empirical fitting of reactor transfer function
        \item Studies can be conducted locally or with cluster resources
    \end{itemize}
\end{frame}

\begin{frame}{Controller Tuning}{MATLAB-Simulink}
    \begin{itemize}
        \item Model control loop in Simulink
        \item Investigate system stability using frequency response tests
        \item Use built-in numerical methods to implement a PID controller tuning method
        \item Repeat for different core conditions to develop gain-schedule and/or look-up table for the controller parameters
    \end{itemize}
\end{frame}

\begin{frame}{Implementation and Testing}{Python}
    \begin{itemize}
        \item Implement control drum reactivity, pre-filter, and PID controller into the process simulation
        \item Test autonomous response to heat exchanger power demand transients
        \item Repeat with controller active
        \item Quantitatively compare response using settling time, dampening ratio, peak overshoot ratio etc.
    \end{itemize}
\end{frame}

\begin{frame}{Timeline}

    \begin{table}
        \centering
        \caption{Timeframe for Execution of Project}
        \begin{tabular}{c|c|c|c|c|c|c|c}
            \textbf{Tasks} & Oct. & Nov. & Dec. & Jan. & Feb. & Mar. & Apr.\\\hline
            \textbf{Control Drums} & \x & \x & \x &  &  &  & \\\hline
            \textbf{Process Simulation} &  & \x & \x & \x &  &  & \\\hline
            \textbf{Controller Tuning} &  &  &  & \x & \x &  & \\\hline
            \textbf{Implementation} &  &  &  &  & \x & \x & \\\hline
            \textbf{Cross-Cutting} &  &  &  &  &  & \x & \x\\\hline
            \textbf{Defend} &  &  &  &  &  &  & \x \\

        \end{tabular}
    \end{table}
        

\end{frame}

\section{Final Remarks}
\begin{frame}{Other Considerations}
    \begin{itemize}
        \item Poison perturbations following power transients \cite{Roberson}
        \item Melting of \textit{in-situ} frozen salt
        \item SCRAM system must be passive
        \item Decay heat system \cite{DecayHeat}
        \item Flow rate control
    \end{itemize}

    \blfootnote{\tiny\cite{Roberson} \tiny\bibentry{Roberson}}
    \blfootnote{\tiny\cite{DecayHeat} \tiny\bibentry{DecayHeat}}
\end{frame}

\begin{frame}{Discussion}{Control Loop}

        \begin{figure}[!ht]
            \centering
            \input{../tikz/SimpleControlLoop}
            \caption{Simplified control loop of a natural circulation \acs{msnb}}
            \label{fig:SimpleControlLoop}
        \end{figure}
    
\end{frame}

\begin{frame}{Acknowledgements}
    \centering
    This work and my coursework is being completed under a Graduate Fellowship funded by \acf{nrc}
\end{frame}


\begin{frame}{References}
    \bibliographystyle{neup}
    \footnotesize
    \bibliography{../References}
\end{frame}


\end{document}
