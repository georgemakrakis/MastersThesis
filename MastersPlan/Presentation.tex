\documentclass[aspectratio=1610,pdftex,dvipsnames]{beamer}
\usetheme{Boadilla}
\usecolortheme{seahorse}
\beamertemplatenavigationsymbolsempty 
\addtobeamertemplate{footnote}{\hskip -2em}{}

\definecolor{BackGround}{RGB}{255,250,240}
\setbeamercolor{background canvas}{bg=BackGround}

\usepackage{caption}
\captionsetup[figure]{labelformat=empty}

%packages and definitions
\usepackage{enumitem}
\setlist[itemize]{label=\textbullet}
\usepackage{amsmath,cancel,nicefrac}
\usepackage{ulem}
\usepackage{graphicx,animate}
%\usepackage{mypythonhighlight}% allows inclusion of graphics
\graphicspath{{../img/}} %path to graphics
\usepackage[yyyymmdd]{datetime} %date format
\renewcommand{\dateseparator}{.}

%%%%%
\usepackage{pgf}
\usepackage{tikz} % required for drawing custom shapes
\usetikzlibrary{shapes,arrows,automata,trees}
%%%%%

\usepackage{booktabs} % nice rules (thick lines) for tables
\usepackage{microtype} % improves typography for PDF

\usepackage[acronym,nomain,nonumberlist]{glossaries}
%\makenoidxglossaries

\usepackage{xcolor,colortbl} %change font color
\usepackage[numbers,sort&compress]{natbib} %use 'numbers' for numbered citations; 'round' for () instead [] for inline citations; nsf.bst
\usepackage{bibentry}


\setlength{\bibsep}{0pt} %sets space between references
%\renewcommand{\bibsection}{} %suppresses large 'references' heading
\renewcommand\bibpreamble{\vspace{-0.2\baselineskip}} %sets spacing after heading if not using default references heading

%%%%% user commands
\newcommand\blfootnote[1]{%
  \begingroup
  \renewcommand\thefootnote{}\footnote{#1}%
  \addtocounter{footnote}{-1}%
  \endgroup
}


\makeatletter
\renewcommand{\@biblabel}[1]{#1.\hfill} %bibliography ordered list has numbers left flush
\makeatother

\newcommand{\edit}[1]{\textcolor{blue}{#1}} %shortcut for changing font color on revised text
\newcommand{\fn}[1]{\footnote{#1}} %shortcut for footnote tag
\newcommand*\sq{\mathbin{\vcenter{\hbox{\rule{.3ex}{.3ex}}}}} %makes a small square as a separator $\sq$
\newcommand{\sk}[1]{\sout{#1}} %shortcut for strikethrough
\newcommand{\x}{\cellcolor{lightgray}\textbf{X}} %use to shade in table cell

\newcommand{\acf}{\acrfull} %full acronym
\newcommand{\acl}{\acrlong} %long acronym
\newcommand{\acs}{\acrshort} %short acronym
\newcommand{\acfp}{\acrfullpl} %full acronym plural
\newcommand{\aclp}{\acrlongpl} %long acronym plural
\newcommand{\acsp}{\acrshortpl} %short acronym plural
\newcommand{\Acf}{\Acrfull} %full acronym first letter capital
\newcommand{\Acl}{\Acrlong} %long acronym first letter capital

\newacronym{pid}{PID}{Proportional-Integral-Derivative}
\newacronym{msnb}{MSNB}{Molten Salt Nuclear Battery}
\newacronym{nrc}{NRC}{Nuclear Regulatory Commission}

\newcommand{\UF}[1][4]{$UF_{#1}$}
\newcommand{\flinak}{$FLiNaK$ }

%%%%%

%%%%%



\title[\acs{pid} Controller Design: \acs{msnb}]{Design of a \acs{pid} Controller for a Molten Salt Microreactor}
\subtitle{Master's Plan}
\author[Root]{Sam J. Root,\textsuperscript{1}\\
    Major Professor: Michael McKellar,\textsuperscript{1}\\
    Committee Members: Robert A. Borrelli\textsuperscript{1}, 
    Dakota Roberson\textsuperscript{2}
    }
\institute[Idaho Falls Center]{\vspace{0.5cm}\\
    University of Idaho $\sq$ Idaho Falls Center for Higher Education\\
    \textsuperscript{1}Department of Nuclear Engineering and Industrial Management\\
    \textsuperscript{2}Department of Electrical and Computer Engineering\\
    \vspace{0.10in}
    %\includegraphics[width=0.20\textwidth]{ne-logo.jpg}
    }

\date{2022.10.13}
    

\AtBeginSection[]{
    \begin{frame}
        \vfill
        \centering
        \begin{beamercolorbox}[sep=8pt,center,shadow=true,rounded=true]{title}
            \usebeamerfont{title}\insertsectionhead\par
        \end{beamercolorbox}
        
        \vfill
        
    \end{frame}
}

\begin{document}
    \nobibliography*
{    \setbeamertemplate{footline}{} 
    \begin{frame}
        \titlepage
    \end{frame}
} 

\begin{frame}{Outline}
    \tableofcontents
\end{frame}

\section{Scope}
\begin{frame}{\acf{msnb}}
\begin{columns}
    %Column 1
    \begin{column}{0.5\textwidth}
        \begin{itemize}
            \item<1-> Self-Contained liquid fueled molten salt micro-reactor
            \item<2-> 1 MW design using \UF dissolved in \flinak
            \item <3-> Criticality is manipulated using axial control drums \begin{itemize}
                \item Neutron absorber plate covering cylinders of neutron reflector
                \item Drums are rotated to point more absorber towards the core to insert negative control reactivity
            \end{itemize}
        \end{itemize}
    \end{column}
    %Column 2
    \begin{column}{0.5\textwidth}
        \only<1-2>{
        \begin{figure}[!ht]
            \centering
            \begin{tikzpicture}
    %Core
    \draw node at (1.5,1.5) {Core};
    \draw[red, very thick] (0,0) rectangle (3,3);
    \filldraw[red,opacity=0.2] (0,0) rectangle (3,3) ;
    %Riser/Chimney
    \draw[->] (1.5,3) -- (1.5,3.5);
    %HEX
    \draw node at (1.5,4) {Heat Exchanger};
    \draw[blue, very thick] (0,3.5) rectangle (3,4.5);
    \filldraw[blue,opacity=0.2] (0,3.5) rectangle (3,4.5) ;
    %Downcomer
    \draw[->] (3,4) -- (3.5,4) -- (3.5,-0.5)  -- (1.5,-0.5);
    \draw[->] (0,4) -- (-0.5,4) -- (-0.5,-0.5)  -- (1.5,-0.5);
    \draw[->] (1.5,-0.5) -- (1.5,0);
\end{tikzpicture}

            \caption{Simplified schematic drawing of an \acs{msnb}}
            \label{fig:tikz_msnb}
        \end{figure}
        }
        \only<3>{

        }
    \end{column}
\end{columns}
\end{frame}

\begin{frame}{Background on \acs{msnb}}
    \begin{block}{Neutronics}
        \cite{PetersonMS}
    \end{block}
    
    \begin{block}{Thermal Hydraulics}
        \cite{CarterPHD}
    \end{block}
    
    \begin{block}{Process Control}
        Me
    \end{block}

    \blfootnote{\tiny\cite{CarterPHD} \tiny\bibentry{CarterPHD}}
    \blfootnote{\tiny\cite{PetersonMS} \tiny\bibentry{PetersonMS}}
\end{frame}

\begin{frame}{\acs{msnb} design}
    Figures from plotter (neutronics paper?), with a focus on control actuation
\end{frame}

\section{Applied Literature Review}

\begin{frame}{Passive Feedback}
    
\end{frame}

\begin{frame}{Main Operational Control Problem - Transport Delay}
    
\end{frame}

\begin{frame}{Time-Variance and Non-Linearity}
    
\end{frame}

\section{Future Work}

\begin{frame}{Control Drum Characterization}{MCNP}
    
\end{frame}

\begin{frame}{Process Simulation}{Python}
    
\end{frame}

\begin{frame}{Controller Tuning}{MATLAB-Simulink}
    
\end{frame}

\begin{frame}{Implementation and Testing}{Python}
    
\end{frame}

\begin{frame}{Timeline}

    \begin{table}
        \centering
        \caption{Timeframe for Execution of Project}
        \begin{tabular}{c|c|c|c|c|c|c|c}
            \textbf{Tasks} & Oct. & Nov. & Dec. & Jan. & Feb. & Mar. & Apr.\\\hline
            \textbf{Control Drums} & \x & \x & \x &  &  &  & \\\hline
            \textbf{Process Simulation} &  & \x & \x & \x &  &  & \\\hline
            \textbf{Controller Tuning} &  &  &  & \x & \x &  & \\\hline
            \textbf{Implementation} &  &  &  &  & \x & \x & \\\hline
            \textbf{Cross-Cutting} &  &  &  &  &  & \x & \x\\\hline
            \textbf{Defend} &  &  &  &  &  &  & \x \\

        \end{tabular}
    \end{table}
        

\end{frame}

\section{Final Remarks}
\begin{frame}{Other Considerations}
    
\end{frame}

\begin{frame}{Discussion}

\end{frame}

\begin{frame}{Acknowledgements}
    \centering
    This work and my coursework is being completed under a Graduate Fellowship funded by \acf{nrc}
\end{frame}


\begin{frame}{References}
    \bibliographystyle{neup}
    \footnotesize
    \bibliography{../References}
\end{frame}


\end{document}
