\chapter{Abstract}
The Molten Salt Nuclear Battery (MSNB) is a Generation IV microreactor concept that employs uranium-tetrafluoride fuel dissolved directly in the primary coolant. This thesis presents two dynamic multi-physics models implemented in Python 3.10 using Euler's forward method. 

The first model is a zero-dimensional, object-oriented nuclide concentration code, intricately linking the mathematics of the xenon-135 decay chain to two-phase equilibrium mass transport. This model was developed to investigate the potential of fission gas stripping in reducing reactor downtime and extending fuel lifetime. Rigorous validation against simplified analytic solutions of the system of ordinary differential equations provides confidence in the accuracy of the model, while subsequent case studies demonstrate the concept to be thermodynamically favorable.

The second model couples the natural-circulation flow mode to reactor point kinetics with the uniform-state uniform-flow condition to simulate the core's response to demand transients. During the investigation, the reactor demonstrated passive stability in both steady-state operation and under moderately aggressive transient conditions. Leveraging this autonomous response, in combination with reactivity actuators characterized through Monte-Carlo neutronics with Serpent 2, a PID feedback control loop was designed. Using the Ziegler-Nichols tuning methodology, the controller reduces the settling time following step-changes to power demand by from 40 minutes to approximately 5 minutes, an order of magnitude improvement, and effectively eliminates set-point overshoot following transients with a ramp-rate of 400 kW/min.

