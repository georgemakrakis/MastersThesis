\chapter{Process Control Engineering}
\label{Chapter:Controls}

There are two main goals in process control engineering:
\begin{enumerate*}
    \item Reference tracking, where a process variable is matched to a set-point which may be changed over time; and 
    \item Disturbance rejection, where the process variable is held to the set-point despite outside influence upsetting it;
\end{enumerate*}
This is usually achieved by a controller which detects the process variable using a sensor/transmitter and controls the process variable by manipulating an actuator. 

\section{Feedback}
The most common type of controller is a feedback controller. The controller takes action based on the `error' ($e$) between the set-point ($SP$) and process-variable ($PV$) (Eqn. \ref{eqn:error}).

\begin{equation}\label{eqn:error}
    e(t) = PV(t) - SP(t)
\end{equation}

The action, or controller output ($u$) is often determined by a \acf{pid} equation (Eqn. \ref{pid}), which considers the instantaneous, cumulative, and predictive error in determining the proper actuation \cite{Bequette}. 

\begin{equation}\label{eqn:pid}
    u(t) = K_P e(t) + K_I \int_0^t e(t)dt + K_D \frac{de(t)}{dt}
\end{equation}

The first term is the proportional control term. The control output is manipulated in proportion to the error defined by the proportional gain constant ($K_P$). A high gain yields an aggressive controller that is prone to overshooting the set-point, while a low gain may result in steady-state offset.  

The second term is the integral control term, which considers the historical cumulative error (calculated by taking the time integral of the error) in an effort to eliminate steady-state offset that a P-Only controller may exhibit. As the process variable settles around the set-point, the cumulative error approaches a constant value and the effect of the integral controller diminishes.

The third term is the derivative control term, which estimates the time rate of change of the error to dampen overshoot. This mechanism, sometimes referred to as anticipatory control, slightly reduces the proportional response to the error when the error is changing rapidly. This results in reducing the peak overshoot. A wel tuned anticipatory gain can allow a more aggressive proportional gain to be used without the large overshoot.

Instead of using three different gain constants, it is common for controllers to be tuned in terms of a single controller gain ($K_C$) plus two time constants: 
\begin{enumerate*}
    \item The integral time constant ($\tau_I$); and
    \item The derivative time constant ($\tau_D$);
\end{enumerate*}
In this case, Equation \ref{eqn:pid} is rewritten as:
\begin{equation}\label{eqn:pid-tau}
    u(t) = K_C \left( e(t) + \tau_I^{-1} \int_0^t e(t)dt + \tau_D \frac{de(t)}{dt}\right)
\end{equation}

\section{Feedforward}
The term `Feedforward' can be used to refer to any element in the control block diagram that exists outside of the feedback loop.

\subsection{Disturbance Feedforward}
Not that useful since disturbance transport delay is on the order of minutes and disturbance dynamics are on the order of milliseconds

\subsection{Pre-Filter}
This could be electronic (less ideal) or physically realized by decoupling 


\section{Time Variance}
Fissile depletion - time function parameters or look-up table to gain-schedule and turn the time variant system into a shift invariant system.

In addition to the relatively slow time variance of fissile fuel depletion during steady-state critical operation, there are specific times in a \acs{msnb}'s expected operational life-cycle that exhibit a higher degree of time variance: 
\begin{enumerate*}
\item Start-up; \item Shut-down; and \item Re-start.
\end{enumerate*}

\subsection{Start-up}
Black-start may need to deal with thawing salt - main concern is fission product neutron poison build-up (discuss the burnable poison stuff)

\subsection{Shut-down}
Planned shut-down

Emergency Shutdown/SCRAM(must be passive)

Decay heat and keeping the salt liquid for restart

\subsection{Re-start}
\Xe stripper

