\chapter{Introduction}
\label{Chapter:Introduction}

The world is attempting to move away fossil fuel as its main energy source \cite{ValluriPHD}. \acf{nrel} has partnered with over 700 organizations, including large manufacturing companies, to de-carbonize supply chains \cite{NREL-partner}. Nuclear power has been well established as an alternative for base-load electrical generation with 93 facilities in the United States 435 globally, most of which generate on the order of 1 GWe rather steadily. There remains a need for smaller reactors to be deployed in more dynamic applications such as small remote grids (e.g. islands, military bases, and off-grid industrial sites), manufacturing, and power-peaking \cite{DoD-remote}. Dow Chemical is working with X-energy to deploy a \acf{smr} at a Gulf Coast facility by 2030 \cite{DowXe}. Even smaller energy utilizers could turn to micro-reactors to fill their needs; to make this a reality, robust control systems for microreactors such as the \acs{msnb} capable of ramping up and down production to fit demand must be designed.

\section{Background}
The \acf{msnb} is a self contained design for a liquid fueled molten salt microreactor \cite{CarterPHD,PetersonMS}. It is fueled by an inorganic form of uranium, \UF, dissolved in a coolant salt such as \flinak (a eutectic mixture of three alkali fluorides) or \flibe  (a mixture of $LiF$ and $BeF_2$) \cite{RoperOverview}. Heat is generated in the core by fission, is transported by the natural circulation of the coolant/fuel salt, and rejected to a secondary working fluid in an integrated heat exchanger. Criticality is manipulated using axial control drums, which may be rotated to aim either a neutron reflecting material or a neutron absorbing material towards the core.

\subsection{Molten Salt Reactors}
Molten salts are highly desirable in high temperature applications due to their excellent thermophysical properties \cite{RoperReview}. Salt mixtures have been developed to have very wide liquid temperature ranges (i.e. low melting point and high vaporization point). They also have very high volumetric heat capacities compared to other high temperature coolants (which tend to be gaseous), and are able to operate at or slightly above ambient pressure. These properties combine to make molten salts excellent choices in heat transfer and thermal storage applications. Furthermore, they are extremely strong electrolytes which cn be useful as solvents, catalysts, or reagents in certain chemical reactions including a pyrometallurgical method for reprocessing spent nuclear fuel \cite{Simpson}.

\acfp{msr} are a family of nuclear reactor in which a fuel salt (containing fissile and/or fertile nuclides) is dissolved in a coolant salt \cite{RoperOverview}. The concept was proven by the \acl{msr} Experiment at \acf{oak} in the 1960s \cite{MSRE}. It has yet to take off beyond the research reactor sector, but it has re-emerged as a Gen-IV reactor concept, with a team at the Shangai Institute of Applied Physics gaining approval to operate a now fully constructed thorium breeding \acs{msr} \cite{china}. Some of the benefits of \acsp{msr} over more conventional \acsp{lwr} include:
\begin{itemize}
    \item Higher operating temperatures allow for use in applications requiring high-grade process heat, and yield higher thermal efficiency \cite{RoperOverview};
    \item Lower pressure operating pressure lends itself to inherant safety, and less expensive (thinner) components \cite{RoperReview};
    \item The ability to burn minor actinides supports the goal of reducing global stockpiles of high-level waste \cite{RoperReview};
    \item Natural circulation of the fuel introduces an additional feedback mechanism that presents the possibility of autonomous load following of certain power demand transients \cite{CarterNumerical};
    \item There is no concern of core melt-down as the reactor is designed for liquid fuel;
    \item The liquid state homogenizes nuclides throughout the core, which minimizes burn-up gradient to produce a flatter temperature and power profile within the core \cite{Lamarsh,TodreasKazimi1}. The flowing nature also allows for online reprocessing, removing fission products and poisons during operation;
\end{itemize}

They also carry some demerits:
\begin{itemize}
    \item Molten salts are very corrosive \cite{RoperRedox};
    \item The chemistry of the coolant (not only the fuel) is constantly changing due to fission, transmutation, and impurities from corrosion;
    \item Lithium is commonplace, so tritium production is unavoidable. Off-gas systems need to be robust to handle tritium as well as radionuclide noble gasses, halides, and interhalides \cite{HolcombOffgas};
\end{itemize}

\subsection{Micro Reactors}
Its like a reactor but smol.

\section{Scope}
As a developing design, work has been done on neutronics \cite{PetersonMS}, thermal-hydraulics and autonomous load following \cite{CarterPHD}, and corrosion concerns \cite{RoperPHD}. However, until now, little to no work has been done on the control system. First and foremost, this work details a multiphysics characterization of the \acs{msnb} required to design a feedback controller capable of matching the core power generation to the secondary power demand. In addition to the main control mode of following power transients during normal operation, specific discussion is centered around more dynamic time periods, namely: 
\begin{enumerate*}[label=\arabic*)]
    \item initial start-up;
    \item shutdown, both planned and emergency; and
    \item restart;
\end{enumerate*}