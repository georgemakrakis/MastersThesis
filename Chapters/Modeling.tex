\chapter{Reactor Characterization}
\label{Chapter:Modeling}
\begin{figure}[!ht]
    \centering
    \resizebox{\textwidth}{!}{
\begin{tikzpicture}
    %Pre-filter
    \draw[->] (-6,0)node[anchor=east]{$\dot{Q}_{HEX}$} -- (-4.5,0);
    \draw (-4.5,-0.5) rectangle (-3.5,0.5) node[pos=0.5]{$F(s)$};
    %Sum
    \draw[->] (-3.5,0) -- (-2,0) node[pos=0.5,anchor=south]{$\dot{Q}_{Core}^{SP}$};
    \draw (-1.75,0) circle (0.25) node{\scriptsize$\textbf{-}$};
    %Controller
    \draw[->] (-1.5,0) -- (-0.5,0)node[pos=0.5,anchor=south]{$e$};
    \draw (-0.5,-0.5) rectangle (0.5,0.5) node[pos=0.5]{$C(s)$};
    \draw[->] (0.5,0) -- (1.5,0) node[pos=0.5,anchor=south]{$u_{CD}$};
    %Actuator
    \draw (1.5,-0.5) rectangle (2.5,0.5) node[pos=0.5]{$A(s)$};
    \draw[->] (2.5,0) -- (3.5,0) node[pos=0.5,anchor=south]{$\rho_{CD}$};
    %Sum
    \draw (3.75,0) circle (0.25) node{\scriptsize$\textbf{+}$};
    %Process
    \draw[->] (4,0) -- (5,0) node[pos=0.5,anchor=south]{$\rho$};
    \draw (5,-0.5) rectangle (6,0.5) node[pos=0.5]{$P(s)$};
    \draw[->] (6,0) -- (8,0) node[anchor=west]{$\dot{Q}_{Core}$};
    %Transducer
    \draw[->] (7,0) -- (7,-1.5) -- (2.5,-1.5);
    \draw (1.5,-2) rectangle (2.5,-1) node[pos=0.5]{$H(s)$} ;
    \draw[->] (1.5,-1.5) -- (-1.75,-1.5) -- (-1.75,-0.25);
    %Passive Feedback
    %Core Feedback
    \draw[->] (7,0) -- (7,2.5);
    \draw (6.5,2.5) rectangle (7.5,3.5) node[pos=0.5]{$G_{C}(s)$};
    \draw[->] (6.5,3) -- (4.25,3);
    %HEX Feedback
    \draw[->] (-5,0) -- (-5,2.5);
    \draw (-4.5,2.5) rectangle (-5.5,3.5) node[pos=0.5]{$G_{H}(s)$};
    \draw[->] (-4.5,3) -- (3.25,3)node[pos=0.5,anchor=south]{$T_{cold}$};
    %Sum
    \draw (3.75,1.5) circle (0.25) node{\scriptsize$\textbf{+}$};
    \draw[->](3.75,1.25) -- (3.75,0.25);
    %TemperatureFeedback
    \draw (1.5,1) rectangle (2.5,2) node[pos=0.5]{$\alpha_T$};
    \draw[->] (2.5,1.5) -- (3.5,1.5) node[pos=0.5,anchor=south]{$\rho_{T}$};
    %FlowFeedback
    \draw (3.25,2.5) rectangle (4.25,3.5) node[pos=0.5]{$\alpha_F$};
    \draw[->] (3.75,2.5) -- (3.75,1.75) node[pos=0.5,anchor=west]{$\rho_{F}$};
    %Downcomer
    \draw[->] (0,3) -- (0,2);
    \draw (-0.5,1) rectangle (0.5,2) node[pos=0.5]{$\theta_{DC}$};
    \draw[->] (0.5,1.5) -- (1.5,1.5);
    %Riser
    \draw[->] (5.375,3) -- (5.375,4.5) -- (0.5,4.5) node[pos=0.5,anchor=south]{$T_{hot}$};
    \draw (-0.5,4) rectangle (0.5,5) node[pos=0.5]{$\theta_{R}$};
    \draw[->] (-0.5,4.5) -- (-5,4.5) -- (-5,3.5);


\end{tikzpicture}
}
    \caption[Control loop of a natural circulation \acs{msnb}]{Control loop of a natural circulation \acs{msnb}. It is a normal feedback loop with a pre-filter, with the addition of the passive feedback mechanisms. The core ($\dot{Q}_{Core}$) and heat exchanger ($\dot{Q}_{HEX}$) powers go through the respective temperature dynamics ($G_C$ and $G_H$) and time delays for the riser ($\theta_R$) and downcomer ($\theta_{DC}$) before being converted to reactivity by the temperature($\alpha_T$) and flow ($\alpha_F$) feedback mechanisms. The passive reactivity feedback is combined with the control drum reactivity ($\rho_{CD}$) and fed into the reactor dynamics ($P(s)$).  }
    \label{fig:ReactorControlLoop}
\end{figure}


\section{Reactor Design selection}
\note{Accident tolerant control drums} \note{helix device breakage safety issue}
\section{Neutronics Modeling}

\section{Process Simulation}