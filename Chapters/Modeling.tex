\chapter{Reactor Characterization}
\label{Chapter:Modeling}
\begin{figure}[!ht]
    \centering
    \resizebox{\textwidth}{!}{\input{tikz/ReactorControlLoop}}
    \caption[Control loop of a natural circulation \acs{msnb}]{Control loop of a natural circulation \acs{msnb}. It is a normal feedback loop with a pre-filter, with the addition of the passive feedback mechanisms. The core ($\dot{Q}_{Core}$) and heat exchanger ($\dot{Q}_{HEX}$) powers go through the respective temperature dynamics ($G_C$ and $G_H$) and time delays for the riser ($\theta_R$) and downcomer ($\theta_{DC}$) before being converted to reactivity by the temperature($\alpha_T$) and flow ($\alpha_F$) feedback mechanisms. The passive reactivity feedback is combined with the control drum reactivity ($\rho_{CD}$) and fed into the reactor dynamics ($P(s)$).  }
    \label{fig:ReactorControlLoop}
\end{figure}


\section{Reactor Design selection}
\note{Accident tolerant control drums} \note{helix device breakage safety issue}
\section{Neutronics Modeling}
Alternating criticality and burn-up modeling - poisoning and burn-up

\subsection{Excess Reactivity and Shutdown Margin}


\subsection{Neutron Spectra}

\section{Process Simulation}