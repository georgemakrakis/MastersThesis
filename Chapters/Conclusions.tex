\chapter{Conclusions}
\label{Chapter:Conclusions}

\section{Limitations}

-Velocity should be damped by inertia... First order vs. second order...


\section{Future Work}\label{Chapter:Conclusions-FutureWork}
\subsection{Fuel Salt Thawing}
Because microreactors are meant to be delivered in a fully or mostly assembled state, it is likely that the \acs{msnb} will be shipped with the molten fuel/coolant salt mixture frozen \textit{in-situ}. The salt will need to be melted before initial start-up, and the heat source for melting the cannot be fission, as the \acs{msnb} requires advective heat removal caused by natural circulation; this is not possible if the flow channels between the core and heat exchanger are frozen. One possible method for salt thawing involves passing low-voltage high-current electricity through the pipes in contact with the salt, similar to how frozen water pipes are thawed \cite{Thawing}; this would be coupled with the introduction of hot secondary coolant into the heat exchanger to provide the necessary hot reservoir.

\subsection{Neutron Source}
A neutron source will be required to start the fission chain reaction after installation, and after any long periods of inactivity. \U[235] undergoes spontaneous fission \cite[Ch. 6]{Faw}, and at high enrichment may be used as the only neutron seed simply by putting the control actuators in a supercritical orientation. More commonly, a dedicated source is used, such as \Ca[252], which undergoes spontaneous fission much more rapidly, or a composite of a strong alpha-emitter (\eg \Pu[238], \Am[241], \Po[210], or \Ra[226]) and \Be[9] which emits a neutron according to \ref{rxn:Be-n} \cite[Ch. 2]{Handbook}. Dedicated neutron seed materials composed of these nuclides could be placed in the core through specialized mechanisms, though the introduction of the seed species as a soluble salt warrants a feasibility analysis.  

\begin{reaction}\label{rxn:Be-n}
    ^{9}Be + \alpha \to {^{12}C} + n + \gamma
\end{reaction}

\subsection{SCRAM System}
The emergency shutdown (\ie SCRAM) system must be passive. In \acsp{lwr}, this is achieved by including large control rods which are actively held out of the core, so that a loss of power results in automatic insertion. Larger \acs{msr} designs may include a SCRAM tank into which the fuel/coolant salt drains in the event of power failure. These systems are often actuated by a freeze plug \cite{FreezePlug} and put the salt in a subcritical orientation by the inclusion of neutron control materials \cite[Ch. 1]{Charit} and high geometric buckling \cite[Ch. 6]{Lamarsh}.

In the \acs{msnb}, the control drums will be actively actuated such that loss of power results in a negative control reactivity insertion of the greatest possible magnitude. Still, a freeze plug SCRAM tank system should also be included to make the system truly fail-safe \note{how to emphasize that this is author opinion}

\subsection{Decay Heat Removal}
Thermal power continues to be released by the decay of radio-nuclides after the fission chain reaction is stopped. For non-emergency shut-down, this heat may be removed by the same heat exchanger used for the secondary loop. There should also be a passive system that rejects decay heat in the event of total power failure, such as a direct contact system which removes heat through the vaporization of sodium \cite{DecayHeat} in the scram tank. This latent heat driven system is ideal because it minimizes the possibility of the salt freezing due to over-cooling.

\subsection{Flow Rate Control}


\section{Summary Remarks}