\chapter{Conclusions}
\label{Chapter:Conclusions}

\section{Limitations}
A number simplifications were made in the development of the neutronics model, including the use of a simplified geometry rather than a rigorous design, and the decoupling of criticality and temperature. Additionally, the burn-up modeling was conducted assuming the uniform dispersion of all species throughout the molten salt. Advanced techniques exist to study the spatial variability of burn-up in the core; these methods should be employed and coupled to diffusion modeling of key fission products to verify this assumption. 

The process simulation is a reduced-order-model that removed certain complexities to ease the solvability of the problem and reduce computation time at the expense of fidelity. Most notably, the flow loop can be reduced to a single dimension and still exhibit behavior in line with expectations. This is likely a reasonable assumption for the well mixed and case modeled in this work. The present simulation also does not consider fluid inertia nor thermal diffusion, each of which would likely dampen some of the fine grain oscillatory behavior of steady-state responses. Finally, the model assumes linear heat rejection from the heat exchanger and heat generation in the core, as well as a lumping of the reactor point kinetics that govern the core's response. The spatial power profiles could easily be reshaped to match rigorous calculations, however, discretizing the reactor point kinetics to the control volume level would likely introduce degrees of freedom requiring additional constraints.

The control loop design in this work is a simplification of what would be required to maximize performance in controlling the \acs{msnb}. As discussed in Section \ref{sec:transducer}, the sensor is expected to be extremely noisy, so the transmitter signal will need to be processed with a second-order low-pass filter or similar control block. The integral controller is capable of some degree of noise rejection, though not at the level that will be required.  The steady-state error and actuator signal should also be processed in an attempt to eliminate the oscillatory behavior of the control drums after certain dynamics. Refining the controller gain and integral time-constant tuning could reduce over-actuation; this will be explored along with the loop shaping techniques discussed above in a publication to follow this thesis. Finally, it was demonstrated in studying the start-up controller that while the temperature of the flow loop is primarily treated as a disturbance that is effected by the power controller, there are certain transients, particularly during start-up, where the temperature profile must be explicitly controlled. For such transients, a more robust \acs{mm} controller must be designed.


\section{Related Systems}\label{Chapter:Conclusions-FutureWork}
This work was focused on the operational power controller for the \acs{msnb}. There are several control systems that exist outside of the control loop of study that are important to bring the reactor online after installation and facilitate accident tolerance and passive safety, or improve the broader \acs{mm} system.

\subsection{Fuel Salt Thawing}
Because microreactors are meant to be delivered in a fully or mostly assembled state, it is likely that the \acs{msnb} will be shipped with the molten fuel/coolant salt mixture frozen \textit{in-situ}. The salt will need to be melted before initial start-up, and the heat source for melting the cannot be fission, as the \acs{msnb} requires advective heat removal caused by natural circulation; this is not possible if the flow channels between the core and heat exchanger are frozen. One possible method for salt thawing involves passing low-voltage high-current electricity through the pipes in contact with the salt, similar to how frozen water pipes are thawed \cite{Thawing}; this would be coupled with the introduction of hot secondary coolant into the heat exchanger to provide the necessary hot reservoir.

\subsection{Neutron Source}
A neutron source will be required to start the fission chain reaction after installation, and after any long periods of inactivity. \U[235] undergoes spontaneous fission \cite[Ch. 6]{Faw}, and at high enrichment may be used as the only neutron seed simply by putting the control actuators in a supercritical orientation. More commonly, a dedicated source is used, such as \Ca[252], which undergoes spontaneous fission much more rapidly, or a composite of a strong alpha-emitter (\eg \Pu[238], \Am[241], \Po[210], or \Ra[226]) and \Be[9] which emits a neutron according to \ref{rxn:Be-n} \cite[Ch. 2]{Handbook}. Dedicated neutron seed materials composed of these nuclides could be placed in the core through specialized mechanisms, though the introduction of the seed species as a soluble salt warrants a feasibility analysis.  

\begin{reaction}\label{rxn:Be-n}
    ^{9}Be + \alpha \to {^{12}C} + n + \gamma
\end{reaction}

\subsection{SCRAM System}
The emergency shutdown (\ie SCRAM) system must be passive. In \acsp{lwr}, this is achieved by including large control rods which are actively held out of the core, so that a loss of power results in automatic insertion. Many \acs{msr} designs may include a SCRAM tank into which the fuel/coolant salt drains in the event of power failure. These systems are often actuated by a freeze plug \cite{FreezePlug}. When power is lost, the active cooling that keeps the plug frozen is lost, and the plug melts. The molten salt then drains into a tank that excludes the possibility of a critical orientation by the inclusion of neutron control materials \cite[Ch. 1]{Charit} and high geometric buckling \cite[Ch. 6]{Lamarsh}.

In the \acs{msnb}, the control drums will be actively actuated such that loss of power results in a negative control reactivity insertion of the greatest possible magnitude. Still, a freeze plug SCRAM tank system should also be included to make the system truly fail-safe.

\subsection{Decay Heat Removal}
Thermal power continues to be released by the decay of radio-nuclides after the fission chain reaction is stopped. For non-emergency shut-down, this heat may be removed by the same heat exchanger used for the secondary loop. There should also be a passive system that rejects decay heat in the event of total power failure, such as a direct contact system which removes heat through the vaporization of sodium \cite{DecayHeat} in the scram tank. This latent heat driven system is ideal because it minimizes the possibility of the salt freezing due to over-cooling. Sodium boils at a higher temperature than \flinak freezes, so heat rejection from the SCRAM tank stops prior to any molten salt phase-change.

\subsection{Flow Rate Control}
The natural circulation flow mode contributes inverse responses and momentary instability to the dynamic response of the \acs{msnb}. There may be times, such as during start-up, where it may be advantageous to manipulate the molten salt flow rate as well as the control drums in controlling the power and temperature profile. This could be accomplished by adjusting the pressure loss coefficient ($\xi$). Valving in high temperature molten salt applications can be difficult. It is therefore proposed that an offset cam drive could be used to plug perforations in the orifice plate, which would allow the flow restriction to be manipulated while keeping sensitive moving parts external.

\section{Summary Remarks}
A \acs{haleu}-\UF fueled \acs{msnb} has been characterized by a series of dynamic multiphysics models as part of work to design a control loop:
\begin{itemize}
    \item An object-oriented numerical \acs{ode} solver, written in Python 3.10, couples fission product formation, radioactive decay, and neutronic transmutation to 2-phase equilibrium mass transport to study the \Xe decay chain and the effect of fission gas stripping on a load-following \acs{msr};
    \item Alternating criticality and burn-up models were developed in Serpent 2 to study the evolution of the control-reactivity curve over the life of the \acs{msnb};
    \item Mass and energy conservation were coupled to reactor point kinetics in a Python 3.10 \acs{cfd} code to predict the flow rate and power of the \acs{msnb};
    \item Time and spatial advancement was carried out using an \acs{fem} that simulates heat and mass advection according to uniform-state uniform-flow;
    \item The power control loop implements a pre-filter to reshape the heat exchanger demand into the core set-point response, and a \acs{pi} controller to compliment the passive feedback mechanisms inherent to the \acs{msnb};
\end{itemize}

Major results and findings include:
\begin{itemize}
    \item The thermodynamics governing fission gas stripping to remove neutron poisons from molten salt fuel are favorable, and investigation of the associated kinetics is warranted;
    \item The \acs{msnb} design studied in this work has an excess reactivity of 4.834\% at hot-standby;
    \item At a relatively low specific power and due to the time the fuel salt spends outside of the core, equilibrium xenon only reduces the excess reactivity by about 100 pcm;
    \item At hot-standby, the shutdown margin if one control drum fails to actuate is 47.19\%, and 0.417\% if only one control drum successfully actuates;
    \item After 8 full-power years, the \acs{msnb} fuel is depleted enough to no longer provide enough excess reactivity for effective operation;
    \item As the fuel is burnt, the control drums must be rotated more to insert the same amount of excess reactivity;
    \item The process simulation begins with core power oscillations of $\pm$ 2\% for steady-state operation, which dampen to within 0.2\% in 30 minutes;
    \item The autonomous response is stable, but significantly under-damped fro 20\% step functions and ramp functions of up to 400 kW/min;
    \item The \acs{pi} controller with set-point pre-filter improves the settling time by an order of magnitude for step functions, and essentially eliminates overshoot for ramp functions;
    \item The \acs{ss} controller is effective for load-following and non-emergency shut-down, but a more robust \acs{mm} control loop that also controls the temperature profile is also needed for start-up;
\end{itemize}